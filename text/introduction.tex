 Este relatório tem por objetivo apresentar as atividades realizadas pelo bolsista Rafael Serapilha Durelli durante o período de Maio/2013 a Fevereiro/2015,
 referente à bolsa de doutorado concebida pela Fundação de Amparo à Pesquisa do Estado de São Paulo (FAPESP) sob o Processo Número 2012/05168-4. 
 
 É importante salientar que o trabalho em questão tem sido desenvolvido no Departamento de Ciências da Computação e Estatística do Instituto de Ciência 
 Matemáticas e de Computação (ICMC) da Universidade de São Paulo (campos São Carlos/SP).  
 Este trabalho se insere no contexto do grupo de pesquisas em Engenheira de Software, sob a orientação do Prof. Dr. Márcio Eduardo Delamaro.
 Além disso, é importante salientar que este trabalho esta sendo executado em colaboração com o grupo de engenharia de software da Universidade Federal de São Carlos (UFSCAR)\footnote{\texttt{http://dc.ufscar.br}}.  
 Mais especificamente em colaboração com o Prof. Dr. Valter Vieira de Camargo\footnote{\texttt{http://buscatextual.cnpq.br/buscatextual/visualizacv.do?id=S819089}},
 o qual tem grande experiência na área de engenharia de software com ênfase no desenvolvimento de frameworks no contexto da programação orientada a aspectos e reuso de software.  Ressalta-se que o bolsista criou um vínculo científico com o \textit{Institut National de Recherche en Informatique et en Automatique} (INRIA),  
onde realizou um ano de doutorado sanduíche sobre orientação do  
 Prof. Dr. Nicolas Anquetil\footnote{\texttt{http://rmod.lille.inria.fr/web/pier/team/Nicolas-Anquetil}} o qual tem grande experiência na área de manutenção e reengenharia de software. O doutorado sanduíche em questão foi realizado em Maio de 2013 até Março de 2014.

Durante o período concernente a este relatório, o bolsista dedicou-se às atividades técnicas requeridas para concretização do seu projeto de doutorado bem com às atividades exigidas pelo Programa de Pós-Graduação em Ciências de Computação e Matemática Computacional do ICMC\footnote{Destaca-se que algumas atividades já foram reportadas no relatório anterior}. Considerando as atividades técnicas previstas no cronograma para o período vigente, todas foram devidamente realizadas ou estão progredindo de acordo com o estipulado, a saber: (\textit{i}) redação da monografia e aprovação no exame de qualificação, (\textit{ii}) implementação da abordagem proposta no exame de qualificação, (\textit{iii}) redação de artigos. O exame de proficiência em língua inglesa, exigido pelo Programa de Pós-Graduação do ICMC, foi devidamente realizado e uma pontuação satisfatória foi obtida.


\subsection{Convenções adotadas neste relatório}

Ao longo deste relatório, \textit{Itálico} é utilizado para dar ênfases, introduzir novos termos e palavras em inglês. \texttt{Typewriter} é utilizado para operador Java, operador da DSL, palavras chaves, nome de métodos, variáveis e URL que aparecem no texto. Símbolos \ding{202}, \ding{203}, \ding{204}, \ding{205} ou \textcircled{a}, \textcircled{b}, \textcircled{c}, \textcircled{d}, são utilizados para chamar a atenção do leitor para informações importantes em figuras e códigos.

\subsection{Estrutura do relário}

Neste relatório enfatiza-se adaptação de um catalogo de refatoração para o metamodelo \textit{Knowledge Discovery Metamodel} (KDM). Além disso, neste relatório também é enfatizado a implementação de uma ferramenta semi-automática que fornecer suporte ao catalogo de refatoração adaptado para o KDM. Em seguida é apresentado a criação de um metamodelo de refatorações, padronizado que contenha características similares ao KDM, ou seja, independente de plataforma e linguagem. E a definição de uma Linguagem Especifica de Domínio (do inglês \textit{Domain-Specific Language} - \textit{DSL}) para auxiliar a instanciação do metamodelo de refatoração.

\textbf{Organização do relatório}
