
Software é um produto que evolui constantemente para satisfazer às necessidades de seus usuários. Para isso é necessário submetê-lo a constantes atividades de manutenção, que podem degradar o código-fonte, tornando-se cada vez mais difícil de mantê-lo uma vez que, na maioria das vezes, o software não é atulizado, bem como a sua a documentação, culminando em uma situação em que a única documentação confiável é o código-fonte. Sistemas com essas características são denominados sistemas legados. Usualmente, sistemas legados são candidatos à reengenharia.


Segundo~\citet{refactImpro} reengenharia no contexto da evolução do software é utilizada para melhorar a qualidade do software, ou seja, melhorar a extensibilidade, modularidade, reusabilidade, complexidade e manutenibilidade. Por exemplo, reengenharia é útil para converter sistemas legados ou códigos deteriorados em unidades mais modularizadas ou até mesmo migrar tais sistemas para diferentes linguagens de programação e paradigmas.


De acordo com~\citet{chikofskyTax}, a reengenharia de software  é necessário para converter o código legado ou deteriorado em um código mais modular e estruturado, ou até mesmo alterar o paradigma de programação. Segundo~\citet{Sneed:2005} mais da metade dos projetos que aplicam reengenharia falham ao lidar com desafios específicos. De acordo com esse autor, tanto a carência de  padronização durante a atividade de reengenharia quanto a falta de apoio computacional efetivo são os principais problemas que acarretam esse grande número de falhas.  

A falta de um processo padronizado de reengenharia usualmente é um problema pois, usualmente tal processo é realizado de forma totalmente \textit{ad hoc}. Além disso, a carência de um apoio computacional efetivo para auxiliar a atividade de reengenharia também é um problema, uma vez que tal atividade não é trivial e requer que várias mudanças sejam realizadas tanto no código-fonte como em outros artefatos, e.g., documentos de requisitos, casos de uso e diagrama de classes. 

% subsection evolu_o_de_software_e_reengenharia (end)