
 Relatório Científico Parcial apresentado à Fundação de Amparo à Pesquisa do Estado de São Paulo (FAPESP) com o objetivo de elucidar as atividades realizadas  
 pelo bolsista Rafael Serapilha Durelli durante o segundo período de vigência da bolsa concebida sob o Processo Número 2012/05168-4. 
 O referido período teve inicio em Maio de 2013 e foi finalizado em Fevereiro de 2015. 
 Além disso, este relatório também descreve as atividades que foram finalizadas, as atividades que estão em andamentos, bem como as atividades a serem realizadas no próximo período. 
%01/05/2013 a 28/02/2015
%
Durante o período de vigência anterior, o bolsista conduziu atividades voltadas principalmente à obtenção de uma ampla visão da literatura científica sobre os principais temas de pesquisa relacionado ao projeto em questão. Para tal, uma revisão sistemática foi conduzida. Os resultados da revisão forneceram evidências que suportam e motivam a realização do projeto proposto. A condução da revisão sistemática bem como os principais resultados foram apresentados no relatório anterior.
%
Vale ressaltar que no segundo período de vigência da bolsa, correspondente a este relatório, cinco principais atividades foram conduzidas. Tais atividades são: (\textit{i}) redação de artigos, (\textit{ii}) adaptação de um catalogo de refatoração para o metamodelo Knowledge Discovery Metamodel (KDM), (\textit{iii}) implementação de uma ferramenta semi-automática que fornecer suporte ao catalogo de refatoração adaptado para o KDM, (\textit{iv}) criação de um metamodelo de refatorações, padronizado que contenha características similares ao KDM, ou seja, independente de plataforma e linguagem, com o principal objetivo de auxiliar o engenheiro de modernização durante a elaboração de refatorações e (\textit{v}) definição de uma Linguagem Especifica de Domínio (do inglês Domain-Specific Language - DSL) para auxiliar a instanciação do metamodelo de refatoração. Esse relatório apresenta as atividades desenvolvidas no período de Maio/2013 a Fevereiro/2015.
%