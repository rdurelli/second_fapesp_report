
De acordo com a \textit{timeLine} apresentado na Seção~\ref{cronograma} é possível salientar algumas atividades previstas para as etapas seguintes. Dessa forma, a seguir são descritas as atividades que o bolsista pretende conduzir durante as próximas etapas.


\begin{enumerate}
	
	\item \textbf{Redação de artigos:} pretende-se continuar produzindo artigos, com o intuito de documentar o progresso do projeto em questão.

	\item \textbf{Implementar Refatorações Arquiteturais:} o objetivo é gerar um KDM em que o pacote \textit{Structure} represente a arquitetura atual, contendo todos os relacionamentos existentes entre seus elementos. Para isso, um algoritmo deverá analisar o KDM modificado para identificar todos os relacionamentos existentes entre elementos de mais baixo nível (classes, interfaces, métodos) e criar relacionamentos de mais alto nível entre os elementos arquiteturais. As atividades de refatoração possuem então a responsabilidade de transformar esse KDM com o objetivo de solucionar os problemas arquiteturais encontrados. No caso das refatorações específicas de arquitetura, uma atividade que deve ser feita é a migração de elementos do código-fonte entre elementos arquiteturais, como camadas, componentes e subsistemas. Por exemplo, movendo uma classe de uma pacote para outro.


	\item \textbf{Estudo Experimental:} após a implementação da abordagem proposta pelo bolsista, pretende-se avaliar o ambiente de modernização resultante. Pretende-se realizar dois tipos de experimentos. O primeiro consiste em realizar um conjunto de estudos de casos para investigar a viabilidade da abordagem proposta, bem como avaliar o uso das funcionalidades do apoio computacional para fornecer suporte à modernização de sistemas legados. O segundo experimento consistem em realizar avaliações controladas utilizando a metodologia experimental~\cite{Wohlin}, a fim de avaliar o impacto da abordagem proposta, bem como do apoio computacional relacionado a eficiência e impacto das equipes e também a qualidade em termos de modularidade, reuso e manutenibilidade dos sistema resultantes durante a atividade de modernização.

	\item \textbf{Evolução ferramental:} realizar melhorias nas ferramentas para aumentar a automatização da abordagem em questão. Resultados do estudo experimental podem auxiliar a elicitar as limitações e sugerir tais melhorias.

	\item \textbf{Redação da Tese.}

	\item \textbf{Defesa do Doutorado.}  

	 
\end{enumerate}