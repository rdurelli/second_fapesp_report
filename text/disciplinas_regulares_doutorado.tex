
De acordo com o tema central do projeto, foram escolhidas algumas disciplinas julgadas importantes para o bom aproveitamento do trabalho proposto no cronograma inicial. O quadro de disciplina cursadas pelo bolsista é mostrado a seguir:
% subsection disciplinas_regulares_do_curso_de_doutorado_do_icmc_usp (end)

\begin{enumerate}
	\item Engenharia de Software Experimental
		\begin{itemize}
			\item Carga Horária - 90
			\item Créditos: 6 (seis)
			\item Conceito Obtido: A
		\end{itemize}
	\item Revisão Sistemática em Engenharia de Software
		\begin{itemize}
			\item Carga Horária - 90
			\item Créditos: 6 (seis)
			\item Conceito Obtido: A
		\end{itemize}
	\item Validação e Teste de Software
		\begin{itemize}
			\item Carga Horária - 180
			\item Créditos: 12 (doze)
			\item Conceito Obtido: A
		\end{itemize}
	\item Preparação Pedagógica
		\begin{itemize}
			\item Carga Horária - 60
			\item Créditos: 4 (quatro)
			\item Conceito Obtido: A 
		\end{itemize}
	\item Especificação formal de software
		\begin{itemize}
			\item Carga Horária - 180
			\item Créditos: 12 (doze)
			\item Conceito Obtido: A
		\end{itemize}
\end{enumerate}

De acordo com as regras do Programa de Pós-Graduação em Ciências de Computação e Matemática Computacional do ICMC os alunos devem obter uma quantidade mínima de créditos para realizar determinadas tarefas, tais como, realizar o exame de qualificação e entregar a dissertação/tese. Nesse contexto, vale ressaltar que o bolsista em questão com o comprimento das disciplinas listadas a cima obteu-se um total de 40 créditos, sendo apto para realizar o exame de qualificação que exigi o mínimo de 24 créditos.%, bem como 36 créditos são necessários para o depósito da tese. 

Maiores informações sobre o desempenho do bolsista em cada uma das disciplinas e os conceitos obtidos podem ser avaliados mais cuidadosamente no histórico escolar deste relatório (ver Apêndice A).

%Conforme mencionado, durante o período deste relatório, o bolsista realizou tanto atividades técnicas requeridas para a concretização do seu projeto de doutorado quanto atividades exigidas pelo Programa de Pós-Graduação. Neste contexto, nesta subseção são apresentas e detalhadas as atividades técnicas realizadas até o momento. A seguir, cada uma das atividades realizadas é descrita.