

Reengenharia de Software e MDD são duas abordagens diferentes e que foram pesquisadas separadamente durante anos. Porém, recentemente pesquisadores identificam interesses análogos entre ambas abordagens. 

Como dito anteriormente, o principal objetivo da Reengenharia de Software é converter sistemas legados para novos sistemas sem alterar a sua funcionalidade, ou seja, isso implica em entender como o sistema legado foi implementado. Portanto, tal sistema legado deve ser convertido a um nível maior de abstração com o intuito de auxiliar o engenheiro de software a se concentrar apenas em informações importantes desse sistema, nesse ponto que MDD e Reengenharia de Software podem ser utilizados em conjunto.

Nesse contexto, a OMG OMG (\textit{Object Management Group}) propôs a abordagem Modernização Dirigida à Arquitetura (\textit{Architecture-Driven Modernization} - ADM), a qual tem como intuito automatizar e formalizar/padronizar o problema tradicional da Reengenharia de Software, i.e., ADM melhora a abordagem  de reengenharia de software tradicional com a utilização de MDD~\cite{PerezCastillo:2011jo}. Segundo~\citet{rezCastillo:2011gm} ADM é uma padronização definida pela OMG para auxiliar a atividade de reengenharia de sistemas legados, com o diferencial da utilização dos princípios da MDD (ver Seção~\ref{Cap2_Sec2_Desenvolvimento_Dirigido_a_Modelos}). ADM difere das abordagens tradicionais de reengenharia de software por dois principais motivos: (\textit{i}) ADM considera todos os artefatos de um sistema legado como modelos e (\textit{ii}) refatorações do sistema legado são realizadas nos modelos e depois gera-se um novo código-fonte refatorado tendo como base tais modelos. 

De acordo com~\citet{PerezCastillo:2011jo} ADM tem algumas vantagens quando comparada com abordagens de reengenharia convencionais: (\textit{i}) permite que pesquisadores definam técnicas de refatorações independente de linguagem e plataforma; (\textit{ii}) refatorações podem ser definidas como modelos, assim, podem ser reutilizadas, ou seja, utiliza o metamodelo KDM, o qual tem como um dos objetivos aumentar a interoperabilidade entre as ferramentas de reengenharia e (\textit{iii}) com a abordagem ADM é possível criar técnicas de refatorações genéricas e específicas. 

% subsection estudo_detalhado_sobre_moderniza_o_dirigida_arquitetura_ (end)

\begin{itemize}

\item estudar ferramentas para criar uma Linguagem Específica de Domínio: fez-se necessário identificar e estudar algumas ferramentas que auxiliam o desenvolvimento de Linguagem Específica de Domínio. Foram identificadas as seguintes ferramentas, XText\footnote{http://www.eclipse.org/Xtext/}, Eclipse EMF\footnote{http://www.eclipse.org/modeling/emf/} e Eclipse GMF\footnote{http://www.eclipse.org/modeling/gmp/}. Tais ferramentas serão estudadas para o desenvolvimento de um Linguagem Específica de Domínio, a qual será utilizada no terceiro passo da abordagem proposta;

\item Estudo detalhado da abordagem denominada Arquitetura Dirigida a Modelo - (ADM): conforme descrito no projeto submetido anteriormente pretende-se criar uma abordagem para auxiliar o engenheiro de software durante a atividade de reestruturação de sistemas legados. Dessa forma, vez se necessário estudar realizar a modernização de sistemas legados com a utilizanção de 

\end{itemize}
