Hoje em dia diversas companhias estão enfrentando problemas de gerenciamento, manutenção e/ou troca (de parte) de um sistema existente. Esses sistemas geralmente são caracterizados como sistemas legados (SL). SL geralmente são aplicações de grande porte que possuem um papel crucial e importante no contexto de gerenciamento de informação de companhias. Melhorar a compreensão desses SLs (e.g., arquitetura, características, acoplamento, modularização, etc ) é o fator principal durante a evolução/modernização desses sistemas. O processo de obtenção de uma representações de alto nível de um determinado SL é chamado de Engenharia Reversa (ER).

Diferentemente da Engenharia Avante (EA), a ER é comumente definida como o processo de examinar um SL para representa-lo formalmente em um modelo de alto nível de abstração~\citep{Reverse_engineering_and_Design_Recovery_A_Taxonomy}. O principal objetivo para realizar esse processo é para facilitar o entendimento do estado atual do SL. Por exemplo, utilizando esse modelo de alto nível de abstração é possível corrigir possíveis erros, adicionar novas características, reutilizar partes do SL em outros sistemas, ou até mesmo moderniza-lo completamente~\citep{Griffith2011}. De acordo com~\cite{Achievements_and_Challenges_in_Software_Reverse_Engineering}, isto está acontecendo com mais frequência nos dias atuais, em função da necessidade de não só satisfazer novas exigências e expectativas dos usuários, mas também para a adaptação dos SLs para modelos de negócios emergentes, aderindo à mudança da legislação, lidar com a inovação tecnológica (frameworks, \textit{Application Programming Interface} (API), ambientes de desenvolvimento, etc) e ainda preservar a estrutura do sistema para que o mesmo não se deteriore.

Claramente, uma vez que a ER é um processo demorado e sujeito a erros, qualquer solução que auxilie (semi)automaticamente o processo de ER traria ajuda para os engenheiros de software/modernização e, assim, facilitaria sua utilização~\citep{ADMCHAPTERR, ADMBook}. No entanto, essa solução teria de enfrentar vários problemas, a saber: (\textit{i}) heterogeneidade técnica dos sistemas legados; (\textit{ii}) complexidade estrutural destes sistemas legados; (\textit{iii}) escalabilidade da solução desenvolvida; e (\textit{iv}) edaptabilidade.

Na década de 90 algumas pesquisas tinham como intuito desenvolver soluções (semi)automática para auxiliar a ER. No entanto, tais soluções eram focadas em tecnologias orientadas a objectos (OO)~\citep{Software_Reuse_and_Reverse_Engineering_in_Practice}. Também surgiu-se o interesse em processos e ferramentas para auxiliar a compreensão de programas desenvolvidos em OO. Entre muitas propostas, algumas focaram na extração e análise de informações relevantes a partir do código fonte ou componentes de software~\citep{Re2_Reverse_engineering_and_reuse_re_engineering}, enquanto outras focaram em banco de dados relacionais~\citep{An_approach_for_reverse_engineering_of_relational_databases}, código compilado ou arquivos binários~\citep{Reversing_Secrets_of_Reverse_Engineering}, etc. No entanto, estas pesquisas eram bastante específicas para uma tecnologia em particular ou um determinado cenário de ER (por exemplo, migração técnica, análise de software).

Com o surgimento de \textit{Model-Driven Engineering} (MDE)~\citep{Model_Driven_Engineering}, suas diretrizes e técnicas fundamentais tem sido utilizadas para auxiliar a construção de soluções eficazes de ER, ou seja, \textit{Model-Driven Reverse Engineering} (MDRE). MDRE formaliza as representações (modelos) derivadas de SL para garantir um entendimento comum sobre o seu conteúdo. Estes modelos são então utilizados como ponto de partida para a ER. Dessa forma, MDRE beneficia diretamente da extensibilidade, da cobertura, reutilização, integração e automação das tecnologias MDE para fornecer um bom suporte para a ER. No entanto, ainda há uma falta de soluções completas destinadas a cobrir todo o processo de MDRE.

Neste contexto, em 2003 a \textit{Object Management Group} (OMG) criou uma força tarefa para analisar e evoluir os tradicionais processos de ER, formalizando-os e fazendo com que eles fossem totalmente apoiados pelas diretrizes e princípios de MDE. Logo, o termo Modernização Dirigida à Arquitetura (\textit{Architecture-Driven Modernization} - ADM) surgiu como uma solução para os problemas de padronização. A ADM é um processo de modernização de SL que utiliza um conjunto de metamodelos para representar complemente um sistema por meio de diferentes representações arquiteturais. Esses modelos são então submetidos à refatorações e otimizações e o código-fonte é então gerado novamente. Durante a modernização de um sistema são gerados vários modelos de acordo com os metamodelos da ADM, que representam diferentes partes do sistema, como: fluxos de dados, banco de dados, elementos de programação (métodos, classes, tipos de dados, etc.) e arquitetura~\citep{Information_Systems_Transformation_Architecture_Driven_Modernization_Case_Studies, Software_modernization_by_recovering_Web_services_from_legacy_databases}.

O \textit{Knowledge Discovery Metamodel} (KDM) é o principal metamodelo da ADM com uma ampla quantidade de metaclasses, cobrindo desde os níveis mais baixos de abstração de um sistema, como o código-fonte, até níveis mais altos, permitindo a representação de conceitos de qualquer domínio. A idéia principal da ADM é que a comunidade comece a desenvolver ferramentas que atuem somente sobre instâncias do KDM, ao invés de serem dependentes de plataformas e linguagens específicas. Por exemplo, um catálogo de refatorações para o KDM~\citep{iri_catalogue_of_refactoring_2014}\footnote{No periodo de vigência pertinente a esse relatório, um artigo descrevendo um Catalogo de Refatoração Adaptado para o KDM foi publicado em um evento qualis B2 voltado para Engenharia de Software} tem o poder de reestruturar um sistema independentemente da linguagem de programação que foi usada em seu desenvolvimento, uma vez que as refatorações ocorrem em nível do KDM, ou seja, um modelo independente de plataforma e/ou linguagem. 

Sistemas Legados precisam ser refatorados durante toda a sua vida útil para se que adequem a novos requisitos. No entanto, geralmente a má aplicação de refatorações/modernizações em SLs pode causar desvios arquiteturais. Embora refatoração (tanto de baixa granularidade, quanto alta granularidade) seja uma técnica poderosa, e uma atividade recorrente durante a modernização  em SLs, foi constatado durante a condução de um mapeamento sistemático~\citep{iri_systematic_mapping_ADM_2014}\footnote{No periodo de vigência pertinente a esse relatório, um artigo descrevendo um Mapeamento Sistemático foi publicado em um evento qualis B2 voltado para Engenharia de Software} 
que a versão original da ADM, e consequentemente do KDM, não fornecem apoio (por exemplo, a catálogos de refatoração, a metamodelos para definir refatorações, etc) para tal atividade. Além disso, dificilmente a arquitetura de um sistema legado permanece intacta depois de anos de manutenção, isso é, sua arquitetura atual possivelmente é diferente da arquitetura que foi previamente planejada. ADM também não fornece apoio à checagem de conformidade entre a arquitetura planejada e a atual.

Nesse sentido, este relatório enfatiza-se quatro principais atividades, a saber: (\textit{i}) adaptação de um catalogo de refatoração para o metamodelo KDM, (\textit{ii}) implementação de uma ferramenta semi-automática que fornecer suporte ao catalogo de refatoração adaptado para o KDM, (\textit{iii}) criação de um metamodelo de refatorações, padronizado que contenha características similares ao KDM, ou seja, independente de plataforma e linguagem, com o principal objetivo de auxiliar o engenheiro de modernização durante a elaboração de refactorações e (\textit{iv}) definição de uma Linguagem Especifica de Domínio (do inglês Domain-Specific Language - DSL) para auxiliar a instanciação do metamodelo de refatoração. A seção seguinte expõe os principais conceitos que dão embasamento para o entendimento das atividades realizadas pelo outorgado. Na Seção~\ref{sub:refatoracao} é apresentado brevemente o conceito sobre Refatoração, na Seção~\ref{sec:model_driven_development} é descrito os conceitos sobre \textit{Model-Driven Development} (MDD). A Seção~\ref{sec:model_driven_reverse_engineering} brevemente discorre sobre \textit{Model Driven Reverse Engineering} (MDRE), bem como, \textit{Architecture-Driven Modernization} (ADM) e \textit{Knowledge Discovery Metamodel} (KDM) que são os temas guarda-chuva para este projeto. Finalmente, na Seção~\ref{sec:apoio_ferramental} alguns dos apoios ferramentais utilizados nesse projeto são destacados.


Na Seção~\ref{cronograma} são descritas detalhadamente as atividades realizadas pelo outorgado durante o período de vigência da bolsa. Por sua vez, o plano de trabalho para as etapas seguintes são mencionados na Seção Z.
