%% LyX 2.0.1 created this file.  For more info, see http://www.lyx.org/.
%% Do not edit unless you really know what you are doing.
\documentclass[english]{article}
\usepackage[T1]{fontenc}
\usepackage[utf8]{luainputenc}
\usepackage{array}
\usepackage{multirow}

\makeatletter

%%%%%%%%%%%%%%%%%%%%%%%%%%%%%% LyX specific LaTeX commands.
%% Because html converters don't know tabularnewline
\providecommand{\tabularnewline}{\\}

\makeatother

\usepackage{babel}
\begin{document}
\begin{tabular}{|>{\raggedright}p{3cm}|>{\raggedright}p{4cm}|>{\raggedright}p{2cm}|>{\raggedright}p{4cm}|}
\hline 
\multicolumn{4}{|c|}{\textbf{Classificação}}\tabularnewline
\hline 
\hline 
\multicolumn{2}{|c|}{\textbf{Propósito}} & \multicolumn{2}{c|}{\textbf{Escopo}}\tabularnewline
\hline 
\textbf{Criação} & Padrões relacionados à instanciação de objetos. & \textbf{Classe} & Tratam da relação entre classes e suas subclasses. Ilustram uma configuração
estática, definida em tempo de compilação.\tabularnewline
\hline 
\textbf{Estrutural} & Utilizados para composição de classes e objetos. & \multirow{2}{2cm}{\textbf{Objeto}} & \multirow{2}{4cm}{Propõe a relação entre os objetos, podem ser alterados em tempo de
execução.}\tabularnewline
\cline{1-2} 
\textbf{Comportamental} & Caracterizam a forma como as classes ou objetos interagem e distribuem
as responsabilidades. &  & \tabularnewline
\hline 
\end{tabular}
\end{document}
